\documentclass[12pt]{article}
\usepackage[utf8]{inputenc}
\usepackage{amsmath,setspace,geometry}
\usepackage{amsthm}
\usepackage{amsfonts}
\usepackage[shortlabels]{enumitem}
\usepackage{rotating}
\usepackage{pdflscape}
\usepackage{graphicx}
\usepackage{bbm}
\usepackage[dvipsnames]{xcolor}
\usepackage{hyperref}
\hypersetup{colorlinks=true, linkcolor= BrickRed, citecolor = BrickRed, filecolor = BrickRed, urlcolor = BrickRed, hypertexnames = true}
\usepackage[]{natbib} 
\bibpunct[:]{(}{)}{,}{a}{}{,}
\geometry{left = 1.0in,right = 1.0in,top = 1.0in,bottom = 1.0in}
\usepackage[english]{babel}
\usepackage{float}
\usepackage{caption}
\usepackage{subcaption}
\usepackage{booktabs}
\usepackage{pdfpages}
\usepackage{threeparttable}
\usepackage{lscape}
\usepackage{bm}
%\usepackage[top=15truemm]{geometry}
%\usepackage[]{natbib} 
\bibpunct[:]{(}{)}{,}{a}{}{,}
\setlength{\textwidth}{\paperwidth}     % ひとまず紙面を本文領域に
\setlength{\oddsidemargin}{-5.4truemm}  % 左の余白を20mm(=1inch-5.4mm)に
\setlength{\evensidemargin}{-5.4truemm} % 
\addtolength{\textwidth}{-40truemm}     % 右の余白も20mmに
\renewcommand{\baselinestretch}{0.3}
\newtheorem{proposition}{Proposition}

\setcounter{MaxMatrixCols}{20}

\usepackage{setspace}
\setstretch{1.2}
\begin{document}
\title{A Note on Finite Sample Performance of \cite{lange2020beyond}}
\author{Suguru Otani\thanks{\href{mailto:}{suguru.otani@e.u-tokyo.ac.jp}, Market Design Center, University of Tokyo}}
\maketitle

\begin{abstract}
\noindent
%150 words:
XXX

%100 words AER
\textbf{Keywords}: XXX \\
\textbf{JEL code}: XXX
\end{abstract}

\section{Introduction}

\begin{itemize}
    \item Multidimensional Unobservables \cite{matzkin2003nonparametric} appendix A if possible
    \item Nonstationarity
    \item Endogeneity like \cite{borowczyk2013accounting}
    \item Comparison with Cobb Douglass
    \item Numerical results are reported as IMSE like \cite{matzkin2003nonparametric}.
\end{itemize}

\section{Model}
We consider the matching function $m_t(\cdot)$ that maps period-$t$ unemployed $U_t$, per-capita search efficacy/matching efficiency of the unemployed $A_t$, and vacancies $V_t$ into hires $H_t$.
We assume that the underlying data generating process is stationary and that we observe a long enough time-series so that we can treat the joint distribution $G: \mathbb{R}_{+}^3 \rightarrow[0,1]$ of $\left(H_t, U_t, V_t\right)$ as observed. 
Also, denote by $F(A, U)$ the joint distribution of $A$ and $U$.

We identify the matching function as well as unobserved, time-varying matching efficiency, $A .$ 
First, we assume that $V$ and $A$ are independent conditional on $U$, that is, $A \perp V \mid U$. 
Second, we assume that the matching function $m(AU,V):\mathbb{R}_{+}^2 \rightarrow \mathbb{R}$ has constant returns to scale. 
Then, Proposition 1 of \cite{lange2020beyond} proves that $G(H, U, V)$ identifies $F(A, U)$ and $m(A U, V): \mathbb{R}_{+}^2 \rightarrow \mathbb{R}_{+}$ up to a normalization of $A$ at one point of the support of $(A, U, V)$. Their proof is an application of \cite{matzkin2003nonparametric}.\footnote{\cite{borowczyk2013accounting} additionally assume that matching efficiency can be decomposed as follows: $\log A_{t}=\mu + \tau_{t} + \varepsilon_{t}$, where $\mu$ is a constant, $\tau_t$ is a seasonal dummy, and $\epsilon_t$ is an idiosyncratic error.}




\section{Monte Carlo Simulation}
Parameters setting is summarized in Table \ref{tb:parameter_setting}.
For time $t=0,\cdots, T$, we generate the observable and unobservable components as follows.
\begin{itemize}
    \item Generate data on $(U_t,V_t,A_t)$ as the following stationary AR(1) process.
    \begin{align*}
        U_t &= \exp(u_t)\kappa,\quad u_t= \rho^{U} u_{t-1} + \varepsilon_{t}^{U}\\
        V_t &= \exp(v_t)\kappa,\quad v_t= \rho^{V} v_{t-1} + \varepsilon_{t}^{V} \\
        A_t &= \exp(a_t),\quad a_{t}=\rho^{A} a_{t-1} + \varepsilon_{t}^{A}.
    \end{align*}
    \item Generate data on $H_t$ from  $m(A_t U_t,V_t)$ satisfying constant return to scale in Table \ref{tb:parameter_setting}.
    \begin{align*}
        H_t&= m(A_t U_t,V_t)\\
        &=\gamma A_t U_t + \gamma V_t .
    \end{align*}
    \item Determine $(U_0,V_0,A_0)$ at the time $t$ in which $U_t=Median(U)$. 
\end{itemize}


\begin{table}[!htbp]
    \caption{True parameters and distributions}
    \label{tb:parameter_setting}
    \begin{center}
    \subfloat[Parameters]{
    \begin{tabular}{cr}
            \hline
            $\rho^{U},\rho^{V},\rho^{A}$ & $\{0.1,0.2\}$ \\
            $\gamma$ & $0.3$ \\
            \hline
        \end{tabular}
    }
    \subfloat[Distributions]{
    \begin{tabular}{crr}
            \hline
            error on observables&  \\
            $\varepsilon_{t}^{U},\varepsilon_{t}^{V},\varepsilon_{t}^{A}$ & $N(0,1)$  \\
            \hline
            Kernel issue&  \\
            $K(\cdot,\cdot)$ & $Gaussian$  \\
            $h$ & 0.1\\
            \hline
        \end{tabular}
    }
    \subfloat[Specification ]{
    \begin{tabular}{crr}
            \hline
            $m(A_t U_t,V_t)=$ &  \\
            (1) Cobb-Douglas & $(A_t U_t)^{\gamma} V_t^{1-\gamma}$\\
            (2) Perfect substitutes & $\gamma A_t U_t + \gamma V_t$\\
            (3) Fixed proportions & $\min\{\gamma A_t U_t, \gamma V_t\}$ \\
            \hline
        \end{tabular}
    }
    \end{center}
    \footnotesize
    Note: $N:$ Normal distribution. $U:$ Uniform distribution.
\end{table}


\section{Conclusion}


\paragraph{Acknowledgments}



\bibliographystyle{ecca}
\bibliography{matching_function}

\end{document}









